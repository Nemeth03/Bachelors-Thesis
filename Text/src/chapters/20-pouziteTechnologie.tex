\chapter{Použité technológie}\label{ch:technology}

\section{NetworkX}\label{sec:networkx}

NetworkX je open-source knižnica pre Python, ktorá sa zaoberá tvorbou, manipuláciou a analýzou komplexných sietí a grafov.
Táto knižnica podporuje všetky druhy grafov, vrátane orientovaných a neorientovaných grafov, vážených a nevážených grafov, ako aj multigrafov \cite{networkx}.

V tejto práci je knižnica NetworkX využitá na základnú grafovú analýzu.
Na analýzu sietí sú použité rôzne zabudované funkcie, ktoré umožňujú efektívne vypočítať rôzne metriky, ako napríklad stupeň uzlov,
koeficient zhlukovania, priemernú dĺžku najkratšej cesty, hustotu a priemer siete.

Okrem toho je knižnica využitá aj na generovanie rôznych typov bezškálových sietí, ako sú Barabási-Albertov model a Dorogovtsev-Mendesov model.
Pre generovanie daných modelov boli použité zabudované funkcie, $nx.barabasi\_albert\_graph()$ a $nx.dorogovtsev\_goltsev\_mendes\_graph()$,
ktoré umožňujú rýchle, efektívne vytvorenie týchto sietí.

Knižnica NetworkX bola vybraná pre jednoduchosť implementácie, rozsiahlu dokumentáciu a podporu integrácie s inými knižnicami v Pythone, ako sú NumPy a Matplotlib.

Na ukážke \ref{lst:dgm} je zobrazený kód, ktorý generuje Dorogovtsev-Mendesov model za pomoci funkcie $nx.dorogovtsev\_goltsev\_mendes\_graph()$ a následne využije
funkcie $G.number\_of\_nodes()$ pre výpočet počtu vrcholov, $G.number\_of\_edges()$ pre výpočet počtu hrán, $nx.average\_clustering(G)$ pre výpočet koeficientu zhlukovania a
$nx.average\_shortest\_path\_length(G)$ pre výpočet priemernej dĺžky najkratšej cesty v grafe.

\clearpage

\begin{lstlisting}[caption={Ukážka použitia NetworkX pre generovanie DGM modelu a grafovú analýzu.}, label={lst:dgm}]
    import networkx as nx

    G = nx.dorogovtsev_goltsev_mendes_graph(5)
    
    numNodes = G.number_of_nodes()
    numEdges = G.number_of_edges()
    avgClustering = nx.average_clustering(G)
    avgShortestPath = nx.average_shortest_path_length(G)
    
    print(f"Number of nodes: {numNodes}")
    print(f"Number of edges: {numEdges}")
    print(f"Average clustering coefficient: {avgClustering}")
    print(f"Average shortest path length: {avgShortestPath}")
\end{lstlisting}