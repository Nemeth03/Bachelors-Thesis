\chapter*{Záver}
\addcontentsline{toc}{chapter}{Záver}
\markboth{Záver}{Záver}

Hlavným cieľom práce bol návrh a implementácia aplikácie, ktorá umožňuje preskúmať vplyv interpunkčných znamienok na
tvorbu pozičných slovných sietí a ich následnú analýzu. Navrhnutá aplikácia umožňuje používateľovi výber textu, jazyka,
interpunkcie a následne typu analýzy. Podporované sú štyri typy analýz, ktoré sú zobrazované prostredníctvom
tabuliek a grafov: distribúcia stupňov vrcholov, grafová a jazyková analýza štruktúry siete, sledovanie závislosti exponentu
$\gamma$ mocninového rozdelenia od počtu uzlov siete.

Výsledky grafovej analýzy štruktúry siete naznačujú, že interpunkčné znamienka majú vplyv na štruktúru a vlastnosti
slovných sietí. Porovnanie slovných sietí s interpunkciou oproti bez interpunkcie ukázalo, že obsahujú uzly s väčším stupňom a
koeficientom zhlukovania, čo sa prejavilo v ostatných pozorovaných vlastnostiach. Tieto vlastnosti hrali rolu aj v analýze
mocninového rozdelenia, kde sme pozorovali, že exponent $\gamma$ pri distribúcii stupňov vrcholov je nižší pre texty s interpunkciou.

Interpunkčné znamienka hrajú dôležítú úlohu v texte, pretože ovplyvňujú jeho štruktúru a význam. Ich prítomnosť vedie k merateľným
rozdielom pri analýze slovných sietí, najmä v parametroch ako koeficient zhlukovania, stupeň uzlov, dĺžka cesty medzi uzlami a
exponent $\gamma$ mocninového rozdelenia. Na základe získaných výsledkov možno konštatovať, že tieto rozdiely sú konzistentné
a preto je možné cielene zakomponovať interpunkčné znamienka do analýzy textu.

V budúcnosti by bolo možné rozšíriť výskum o ďalšie jazyky, typy textov alebo sa zamerať na iný typ slovných sietí,
ako sú napríklad siete založené na sémantických vzťahoch medzi slovami.

Výsledky tejto práce poskytujú hlbší pohľad na vplyv interpunkcie na štruktúru textu a ukazujú, že môže mať vplyv
štatistické a topologické vlastnosti slovných sietí.