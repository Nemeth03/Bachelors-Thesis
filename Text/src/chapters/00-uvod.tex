\chapter*{Úvod}
\addcontentsline{toc}{chapter}{Úvod}
\markboth{Úvod}{Úvod}

Analýza interpunkčných znamienok v rôznojazyčných textoch predstavuje proces, pri ktorom sa z textu vytvárajú dve samostatné slovné siete.
Prvá sieť obsahuje výhradne slová, zatiaľ čo druhá zahŕňa aj interpunkčné znamienka.
Následne sa tieto dve siete porovnávajú z hľadiska ich štruktúry a vybraných vlastností.

Cieľom tejto práce je preskúmať, ako prítomnosť interpunkčných znamienok ovplyvňuje štruktúru a vlastnosti slovných sietí vytvorených
z textov v anglickom a nemeckom jazyku. Analýza prebieha prostredníctvom rôznych metód, ako sú distribúcia stupňov vrcholov,
grafová aj jazyková analýza štruktúry siete. Práca zároveň prezentuje návrh a implementáciu aplikácie, ktorá umožňuje
výpočet a vizualizáciu týchto analýz pre ľubovoľný text.

V prvej kapitole sa zameriame na teoretický podklad analýzy, ktorý zahŕňa základné pojmy a definície z oblasti teórie grafov,
ako aj prehľad viacerých modelov sietí, rôzne typy slovných sietí a vznik pozičnej slovnej siete. Taktiež sa budeme venovať problematike interpunkcie
v jazyku a jej vplyvu na štruktúru textu.

Druhá kapitola poukazuje na nástroje a technológie, ktoré boli použité pri implementácii aplikácie.

V rámci tretej kapitoly si ukážeme postup riešenia problematiky, návrh a implementáciu jednotlivých komponentov aplikácie,
spôsob jej testovania a možnosti ďalšieho vylepšenia či rozšírenia.

Posledná kapitola sa zaoberá výskumom a porovnaním výsledkov, získaných z analýzy textov s použitím navrhnutej aplikácie.
V tejto práci sa skúmali tri literárne diela v anglickom a nemeckom jazyku, Vianočná koleda a Oliver Twist od Charlesa Dickensa a
Portrét Doriana Graya od Oscara Wildea. Všetky tri diela sú známe a uznávané v literárnej kultúre, pričom sa zameriavajú na rôzne témy a štýly písania.

Na záver sa zameriame na zhodnotenie dosiahnutých výsledkov a prínos tejto práce v oblasti analýzy textu a spracovania prirodzeného jazyka.