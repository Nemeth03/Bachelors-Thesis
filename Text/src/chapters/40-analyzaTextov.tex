\chapter{Analýza textov}\label{ch:textAnalysis}

Výskum sa zameriava na literárne diela dvoch autorov, Charlesa Dickensa a Oscara Wildea v dvoch jazykových variáciách: anglická a nemecká.
Cieľom je preskúmať vplyv interpunkcie na pozičnú slovnú sieť a jej mocninové rozdelenie.
Kapitola je rozdelená do štyroch hlavných častí: vývoj slovnej siete, distribúcia stupňov vrcholov, grafová analýza a jazyková analýza.

\section{Vývoj slovnej siete}\label{sec:vyvojSlovnejSiete}

Ako je možné vidieť na uvedených grafoch obr. č. \ref{fig:growthKoleda}, č. \ref{fig:growthTwist} a č. \ref{fig:growthDorian},
veľkosť exponentu mocninového rozdelenia $\gamma$ sa mení v závislosti od veľkosti siete. Pozorovanie vývoja slovných sietí vytvorených z literárnych diel v rôznych
jazykoch ukázalo, že rozdiel medzi slovnou sieťou bez interpunkcie a slovnou sieťou, ktorá obsahuje interpunkciu je najmä v počiatočnom
štádiu vývoja siete. 

V prípade Charlesa Dickensa sme analyzovali dve literárne diela, Vianočnú koledu a Olivera Twista. V oboch textoch, ako v anglickej, tak aj nemeckej verzii
boli najvýraznejšie zmeny exponentu $\gamma$ pozorované pri raste siete približne do veľkosti 1000 až 2000 vrcholov. Po prekročení tejto hranice sa hodnota exponentu postupne stabilizuje
a pohybuje sa v rozpätí $2$ až $2.2$.

Dielo Portrét Doriana Graya od Oscara Wildea bola analyzovaná rovnako v anglickej a nemeckej verzii. Aj v tomto prípade nastala najväčšia zmena hodnoty exponentu nastala pri raste siete do
veľkosti 1500 až 2000 vrcholov. Následne sa hodnota exponentu postupne stabilizovala a pohybovala sa v rovnakom rozpätí $2$ až $2.2$.

Na základe tejto analýzy boli identifikované dostatočne dlhé úseky textov, ktoré slúžili ako vstupné dáta pre nasledujúce analýzy.

\begin{figure}[htbp]
    \centering
    \begin{subfigure}[b]{0.9\textwidth}
        \includegraphics[width=\textwidth]{images/Growth/Screenshot_13.png}
    \end{subfigure}

    \vspace{0.3cm}

    \begin{subfigure}[b]{0.9\textwidth}
        \includegraphics[width=\textwidth]{images/Growth/Screenshot_14.png}
    \end{subfigure}
    
    \vspace{0.3cm}
    \caption{Zmena veľkosti exponentu $\gamma$ v závisloti od veľkosti siete, text Vianočná koleda od Charlesa Dickensa.}\label{fig:growthKoleda}
\end{figure}

\begin{figure}[htbp]
    \centering
    \begin{subfigure}[b]{0.9\textwidth}
        \includegraphics[width=\textwidth]{images/Growth/Screenshot_17.png}
    \end{subfigure}

    \vspace{0.3cm}

    \begin{subfigure}[b]{0.9\textwidth}
        \includegraphics[width=\textwidth]{images/Growth/Screenshot_18.png}
    \end{subfigure}
    
    \vspace{0.3cm}
    \caption{Zmena veľkosti exponentu $\gamma$ v závisloti od veľkosti siete, text Oliver Twist od Charlesa Dickensa.}\label{fig:growthTwist}
\end{figure}

\begin{figure}[htbp]
    \centering
    \begin{subfigure}[b]{0.9\textwidth}
        \includegraphics[width=\textwidth]{images/Growth/Screenshot_15.png}
    \end{subfigure}

    \vspace{0.3cm}

    \begin{subfigure}[b]{0.9\textwidth}
        \includegraphics[width=\textwidth]{images/Growth/Screenshot_16.png}
    \end{subfigure}
    
    \vspace{0.3cm}
    \caption{Zmena veľkosti exponentu $\gamma$ v závisloti od veľkosti siete, text Portrét Doriana Graya od Oscara Wildea.}\label{fig:growthDorian}
\end{figure}

\clearpage

\section{Grafová analýza}\label{sec:grafovaAnalyza}

Grafová analýza textu pre slovnú sieť bez interpunkcie a so zohľadnením interpunkcie bola vykonaná na rovnakých vzorkách textov.
Vo výsledkoch (tabuľka č. \ref{tab:anglickyBezInterpunkcie}, č. \ref{tab:anglickySInterpunkciou}, č. \ref{tab:nemeckyBezInterpunkcie} a č. \ref{tab:nemeckySInterpunkciou})
jednotlivých analýz môžeme pozorovať niekoľko zmien.

Ako si môžeme všimnúť, rozdiely sa opakovane prejavujú nezávisle od konkrétneho textu a jazyka. Rozdiely medzi slovnými sieťami bez interpunkcie a so zohľadnením interpunkcie
majú konzistentný charakter, pričom sa opakovane prejavuje rovnaký trend zmien v sledovaných parametroch.

Počet vrcholov a počet hrán sa v prípade oboch sietí mení len minimálne, zatiaľ čo maximálny stupeň v slovnej sieti s interpunkciou je výrazne vyšší, pretože
interpunkčné znamienka sú v porovnaní so slovami veľmi časté a vytvárajú tak väčší počet hrán. Napríklad čiarka alebo bodka sa môže spájať s mnohými slovami, čím vznikajú
uzly s vysokým stupňom. Vďaka tomu sa zvyšuje aj priemerný stupeň, ktorý je v prípade slovnej siete s interpunkciou vyšší, okrem textu Portrét Doriana Graya v angličtine, kde sa
priemerný stupeň znížil. 

Priemerný koeficient zhlukovania je v prípade slovnej siete s interpunkciou vyšší, čo naznačuje, že uzly sú v tejto sieti viac zoskupené do tzv. hubov. To vplýva aj na 
priemernú najkratšiu cestu, ktorá je v prípade slovnej siete s interpunkciou nižšia.

Ďalšie zaujímavé pozorovania sa týkajú korelácie medzi rôznymi typmi centrality:
\begin{itemize}
    \item Korelácia medzi stupňom a blízkosťou pohybujúca sa v intervale $0.34$ až $0.50$, ktorá udáva, ako blízko sú jednotlivé uzly.
          V prípade slovnej siete bez interpunkcie je vyššia.
    \item Korelácia medzi stupňom a medziľahlosťou je v oboch prípadoch veľmi vysoká v rozsahu $0.93$ až $0.96$ s minimálnou zmenou a naznačuje, že uzly s vysokým stupňom
          sú často aj uzlami, ktoré prepájajú rôzne časti siete.
    \item Korelácia medzi blízkosťou a medziľahlosťou je v prípade slovnej siete bez interpunkcie opäť vyššia, no vo všetkých prípadoch sa
          pohybuje v rozmedzí $0.23$ až $0.37$, čo naznačuje, že uzly majú slabú pozitívnu koreláciu medzi týmito dvoma typmi centrality. Tento výlsedok naznačuje, že uzly, ktoré sú blízko seba,
          nemusia zohrávať významnú úlohu pri tvorbe najkratších ciest v sieti.
\end{itemize}


\clearpage

\begin{table}[]
\centering
\scriptsize
\begin{tabular}{|c|c|c|c|}
\hline
\multicolumn{4}{|c|}{Grafová analýza, Anglický jazyk, bez interpunkcie} \\ \hline
\textbf{Text} &
  \begin{tabular}[c]{@{}c@{}}Vianočná koleda\\ $\sim$21000 slov\end{tabular} &
  \begin{tabular}[c]{@{}c@{}}Oliver Twist\\ $\sim$22000 slov\end{tabular} &
  \begin{tabular}[c]{@{}c@{}}Portrét Doriana Graya\\ $\sim$29000 slov\end{tabular} \\ \hline
Počet vrcholov & 3725 & 3842 & 3751 \\ \hline
Počet hrán & 14164 & 14762 & 17315 \\ \hline
\begin{tabular}[c]{@{}c@{}}Maximálny\\ stupeň\end{tabular} & 942 & 961 & 942 \\ \hline
\begin{tabular}[c]{@{}c@{}}Minimálny\\ stupeň\end{tabular} & 1 & 1 & 1 \\ \hline
\begin{tabular}[c]{@{}c@{}}Priemerný\\ stupeň\end{tabular} & 7.60483 & 7.68454 & 9.23220 \\ \hline
Hustota siete & 0.00204 & 0.00200 & 0.00246 \\ \hline
\begin{tabular}[c]{@{}c@{}}Korelácia centrality\\ (stupeň, blízkosť)\end{tabular} & 0.43094 & 0.42376 & 0.48416 \\ \hline
\begin{tabular}[c]{@{}c@{}}Korelácia centrality\\ (stupeň, medziľahlosť)\end{tabular} & 0.95007 & 0.95728 & 0.93399 \\ \hline
\begin{tabular}[c]{@{}c@{}}Korelácia centrality\\ (blízkosť, medziľahlosť)\end{tabular} & 0.29077 & 0.29174 & 0.31290 \\ \hline
\begin{tabular}[c]{@{}c@{}}Priemerný\\ koeficient\\ zhlukovania\end{tabular} & 0.35024 & 0.33708 & 0.34747 \\ \hline
\begin{tabular}[c]{@{}c@{}}Priemerná\\ najkratšia\\ cesta\end{tabular} & 2.91347 & 2.91189 & 2.86155 \\ \hline
Priemer & 7 & 8 & 7 \\ \hline
\end{tabular}
\caption{Grafová analýza anglických textov bez interpunkcie}\label{tab:anglickyBezInterpunkcie}
\end{table}

\begin{table}[]
\centering
\scriptsize
\begin{tabular}{|c|c|c|c|}
\hline
\multicolumn{4}{|c|}{Grafová analýza, Anglický jazyk, s interpunkciou} \\ \hline
\textbf{Text} &
  \begin{tabular}[c]{@{}c@{}}Vianočná koleda\\ $\sim$21000 slov\end{tabular} &
  \begin{tabular}[c]{@{}c@{}}Oliver Twist\\ $\sim$22000 slov\end{tabular} &
  \begin{tabular}[c]{@{}c@{}}Portrét Doriana Graya\\ $\sim$29000 slov\end{tabular} \\ \hline
Počet vrcholov & 3737 & 3855 & 3761 \\ \hline
Počet hrán & 14389 & 14820 & 16934 \\ \hline
\begin{tabular}[c]{@{}c@{}}Maximálny\\ stupeň\end{tabular} & 1339 & 1215 & 1214 \\ \hline
\begin{tabular}[c]{@{}c@{}}Minimálny\\ stupeň\end{tabular} & 1 & 1 & 1 \\ \hline
\begin{tabular}[c]{@{}c@{}}Priemerný\\ stupeň\end{tabular} & 7.70083 & 7.68872 & 9.00505 \\ \hline
Hustota siete & 0.00206 & 0.00199 & 0.00239 \\ \hline
\begin{tabular}[c]{@{}c@{}}Korelácia centrality\\ (stupeň, blízkosť)\end{tabular} & 0.41137 & 0.40580 & 0.42678 \\ \hline
\begin{tabular}[c]{@{}c@{}}Korelácia centrality\\ (stupeň, medziľahlosť)\end{tabular} & 0.95424 & 0.96189 & 0.95068 \\ \hline
\begin{tabular}[c]{@{}c@{}}Korelácia centrality\\ (blízkosť, medziľahlosť)\end{tabular} & 0.27621 & 0.28199 & 0.27787 \\ \hline
\begin{tabular}[c]{@{}c@{}}Priemerný\\ koeficient\\ zhlukovania\end{tabular} & 0.44544 & 0.41362 & 0.44917 \\ \hline
\begin{tabular}[c]{@{}c@{}}Priemerná\\ najkratšia\\ cesta\end{tabular} & 2.77038 & 2.80452 & 2.73230 \\ \hline
Priemer & 6 & 6 & 7 \\ \hline
\end{tabular}
\caption{Grafová analýza, Anglický jazyk, s interpunkciou}\label{tab:anglickySInterpunkciou}
\end{table}

\begin{table}[]
\centering
\scriptsize
\begin{tabular}{|c|c|c|c|}
\hline
\multicolumn{4}{|c|}{Grafová analýza, Nemecký jazyk, bez interpunkcie} \\ \hline
\textbf{Text} &
  \begin{tabular}[c]{@{}c@{}}Vianočná koleda\\ $\sim$16000 slov\end{tabular} &
  \begin{tabular}[c]{@{}c@{}}Oliver Twist\\ $\sim$16000 slov\end{tabular} &
  \begin{tabular}[c]{@{}c@{}}Portrét Doriana Graya\\ $\sim$17000 slov\end{tabular} \\ \hline
Počet vrcholov & 3745 & 4133 & 3802 \\ \hline
Počet hrán & 12037 & 12287 & 12631 \\ \hline
\begin{tabular}[c]{@{}c@{}}Maximálny\\ stupeň\end{tabular} & 771 & 731 & 577 \\ \hline
\begin{tabular}[c]{@{}c@{}}Minimálny\\ stupeň\end{tabular} & 1 & 1 & 1 \\ \hline
\begin{tabular}[c]{@{}c@{}}Priemerný\\ stupeň\end{tabular} & 6.42830 & 5.94580 & 6.64440 \\ \hline
Hustota siete & 0.00172 & 0.00144 & 0.00175 \\ \hline
\begin{tabular}[c]{@{}c@{}}Korelácia centrality\\ (stupeň, blízkosť)\end{tabular} & 0.44845 & 0.43471 & 0.50290 \\ \hline
\begin{tabular}[c]{@{}c@{}}Korelácia centrality\\ (stupeň, medziľahlosť)\end{tabular} & 0.94554 & 0.95769 & 0.94727 \\ \hline
\begin{tabular}[c]{@{}c@{}}Korelácia centrality\\ (blízkosť, medziľahlosť)\end{tabular} & 0.31412 & 0.31849 & 0.36585 \\ \hline
\begin{tabular}[c]{@{}c@{}}Priemerný\\ koeficient\\ zhlukovania\end{tabular} & 0.21771 & 0.19410 & 0.22756 \\ \hline
\begin{tabular}[c]{@{}c@{}}Priemerná\\ najkratšia\\ cesta\end{tabular} & 3.20136 & 3.25713 & 3.19821 \\ \hline
Priemer & 9 & 9 & 7 \\ \hline
\end{tabular}
\caption{Grafová analýza, Nemecký jazyk, bez interpunkcie}\label{tab:nemeckyBezInterpunkcie}
\end{table}

\begin{table}[]
\centering
\scriptsize
\begin{tabular}{|c|c|c|c|}
\hline
\multicolumn{4}{|c|}{Grafová analýza, Nemecký jazyk, s interpunkciou} \\ \hline
\textbf{Text} &
  \begin{tabular}[c]{@{}c@{}}Vianočná koleda\\ $\sim$16000 slov\end{tabular} &
  \begin{tabular}[c]{@{}c@{}}Oliver Twist\\ $\sim$16000 slov\end{tabular} &
  \begin{tabular}[c]{@{}c@{}}Portrét Doriana Graya\\ $\sim$17000 slov\end{tabular} \\ \hline
Počet vrcholov & 3757 & 4145 & 3811 \\ \hline
Počet hrán & 12177 & 12546 & 12710 \\ \hline
\begin{tabular}[c]{@{}c@{}}Maximálny\\ stupeň\end{tabular} & 1271 & 1239 & 1060 \\ \hline
\begin{tabular}[c]{@{}c@{}}Minimálny\\ stupeň\end{tabular} & 1 & 1 & 1 \\ \hline
\begin{tabular}[c]{@{}c@{}}Priemerný\\ stupeň\end{tabular} & 6.48230 & 6.05356 & 6.67017 \\ \hline
Hustota siete & 0.00173 & 0.00146 & 0.00175 \\ \hline
\begin{tabular}[c]{@{}c@{}}Korelácia centrality\\ (stupeň, blízkosť)\end{tabular} & 0.34772 & 0.35362 & 0.39363 \\ \hline
\begin{tabular}[c]{@{}c@{}}Korelácia centrality\\ (stupeň, medziľahlosť)\end{tabular} & 0.95683 & 0.96393 & 0.95394 \\ \hline
\begin{tabular}[c]{@{}c@{}}Korelácia centrality\\ (blízkosť, medziľahlosť)\end{tabular} & 0.23112 & 0.24623 & 0.26092 \\ \hline
\begin{tabular}[c]{@{}c@{}}Priemerný\\ koeficient\\ zhlukovania\end{tabular} & 0.34032 & 0.30007 & 0.31879 \\ \hline
\begin{tabular}[c]{@{}c@{}}Priemerná\\ najkratšia\\ cesta\end{tabular} & 2.95941 & 3.03028 & 2.99317 \\ \hline
Priemer & 7 & 7 & 7 \\ \hline
\end{tabular}
\caption{Grafová analýza, Nemecký jazyk, s interpunkciou}\label{tab:nemeckySInterpunkciou}
\end{table}

\clearpage

\section{Distribúcia stupňov vrcholov}\label{sec:distribuciaStupnovvrcholov}



pozorovanie distribucie stupnov vrcholov charakteristiky, porovnanie s a bez interpunkcie, rozne jazyky

\begin{figure}[htbp]
    \centering
    \begin{subfigure}[b]{0.9\textwidth}
        \includegraphics[width=\textwidth]{images/lbdegdist/Screenshot_1.png}
    \end{subfigure}

    \vspace{0.3cm}

    \begin{subfigure}[b]{0.9\textwidth}
        \includegraphics[width=\textwidth]{images/lbdegdist/Screenshot_2.png}
    \end{subfigure}
    
    \vspace{0.3cm}
    \caption{Zmena veľkosti exponentu $\gamma$ v závisloti od veľkosti siete, text Vianočná koleda od Charlesa Dickensa.}
\end{figure}

\begin{figure}[htbp]
    \centering
    \begin{subfigure}[b]{0.9\textwidth}
        \includegraphics[width=\textwidth]{images/lbdegdist/Screenshot_5.png}
    \end{subfigure}

    \vspace{0.3cm}

    \begin{subfigure}[b]{0.9\textwidth}
        \includegraphics[width=\textwidth]{images/lbdegdist/Screenshot_6.png}
    \end{subfigure}
    
    \vspace{0.3cm}
    \caption{Zmena veľkosti exponentu $\gamma$ v závisloti od veľkosti siete, text Oliver Twist od Charlesa Dickensa.}
\end{figure}

\begin{figure}[htbp]
    \centering
    \begin{subfigure}[b]{0.9\textwidth}
        \includegraphics[width=\textwidth]{images/lbdegdist/Screenshot_3.png}
    \end{subfigure}

    \vspace{0.3cm}

    \begin{subfigure}[b]{0.9\textwidth}
        \includegraphics[width=\textwidth]{images/lbdegdist/Screenshot_4.png}
    \end{subfigure}
    
    \vspace{0.3cm}
    \caption{Zmena veľkosti exponentu $\gamma$ v závisloti od veľkosti siete, text Portrét Doriana Graya od Oscara Wildea.}
\end{figure}

\clearpage

\section{Jazyková analýza}\label{sec:jazykovaAnalyza}

tabulky vypocitanych hodnot, porovnanie s a bez interpunkcie, rozne jazyky

\clearpage

\begin{table}[]
\centering
\scriptsize
\begin{tabular}{|c|c|c|c|}
\hline
\multicolumn{4}{|c|}{Jazyková analýza, Anglický jazyk} \\ \hline
\textbf{Veľkosť textu (slová)} &
  \begin{tabular}[c]{@{}c@{}}Vianočná koleda\\ $\sim$21000 slov\end{tabular} &
  \begin{tabular}[c]{@{}c@{}}Oliver Twist\\ $\sim$22000 slov\end{tabular} &
  \begin{tabular}[c]{@{}c@{}}Portrét Doriana Graya\\ $\sim$29000 slov\end{tabular} \\ \hline
Počet interpunkčných znamienok & 4377 & 4350 & 5782 \\ \hline
Najväčší výskyt interpunkčného znamienka & 2055 & 2123 & 2302 \\ \hline
Najmenší výskyt interpunkčného znamienka & 47 & 38 & 16 \\ \hline
Počet unikátnych slov & 3737 & 3855 & 3761 \\ \hline
Maximálna dĺžka slova & 15 & 15 & 16 \\ \hline
Minimálna dĺžka slova & 1 & 1 & 1 \\ \hline
Priemerná dĺžka slova & 6.45063 & 6.73281 & 6.56076 \\ \hline
Počet viet & 1329 & 1425 & 2612 \\ \hline
Maximálna dĺžka vety & 136 & 209 & 124 \\ \hline
Minimálna dĺžka vety & 1 & 1 & 1 \\ \hline
Priemerná dĺžka vety & 15.99473 & 15.80561 & 11.38055 \\ \hline
Počet dvojíc slov & 15057 & 15363 & 17700 \\ \hline
Maximálna frekvencia dvojice & 400 & 309 & 618 \\ \hline
Minimálna frekvencia dvojice & 1 & 1 & 1 \\ \hline
Priemerná frekvencia dvojice & 1.78409 & 1.83616 & 2.09960 \\ \hline
Počet trojíc slov & 22956 & 23647 & 29482 \\ \hline
Maximálna frekvencia trojice & 73 & 119 & 256 \\ \hline
Minimálna frekvencia trojice & 1 & 1 & 1 \\ \hline
Priemerná frekvencia trojice & 1.17015 & 1.19288 & 1.26050 \\ \hline
\end{tabular}
\caption{Jazyková analýza, Anglický jazyk}
\end{table}


\begin{table}[]
\centering
\scriptsize
\begin{tabular}{|c|c|c|c|}
\hline
\multicolumn{4}{|c|}{Jazyková analýza, Nemecký jazyk} \\ \hline
\textbf{Veľkosť textu (slová)} &
  \begin{tabular}[c]{@{}c@{}}Vianočná koleda\\ $\sim$16000 slov\end{tabular} &
  \begin{tabular}[c]{@{}c@{}}Oliver Twist\\ $\sim$16000 slov\end{tabular} &
  \begin{tabular}[c]{@{}c@{}}Portrét Doriana Graya\\ $\sim$17000 slov\end{tabular} \\ \hline
Počet interpunkčných znamienok & 2983 & 2973 & 3224 \\ \hline
Najväčší výskyt Interpunkčného znamienka & 1703 & 1606 & 1580 \\ \hline
Najmenší výskyt Interpunkčného znamienka & 8 & 24 & 25 \\ \hline
Počet unikátnych slov & 3757 & 4145 & 3811 \\ \hline
Maximálna dĺžka slova & 23 & 26 & 24 \\ \hline
Minimálna dĺžka slova & 1 & 1 & 1 \\ \hline
Priemerná dĺžka slova & 7.58957 & 7.91870 & 7.89766 \\ \hline
Počet viet & 1035 & 1041 & 1417 \\ \hline
Maximálna dĺžka vety & 126 & 90 & 114 \\ \hline
Minimálna dĺžka vety & 1 & 1 & 1 \\ \hline
Priemerná dĺžka vety & 15.49758 & 15.25456 & 11.93719 \\ \hline
Počet dvojíc slov & 12703 & 13039 & 13229 \\ \hline
Maximálna frekvencia dvojice & 188 & 163 & 204 \\ \hline
Minimálna frekvencia dvojice & 1 & 1 & 1 \\ \hline
Priemerná frekvencia dvojice & 1.56050 & 1.50732 & 1.58644 \\ \hline
Počet trojíc slov & 17742 & 17706 & 18599 \\ \hline
Maximálna frekvencia trojice & 87 & 53 & 155 \\ \hline
Minimálna frekvencia trojice & 1 & 1 & 1 \\ \hline
Priemerná frekvencia trojice & 1.11724 & 1.10996 & 1.12834 \\ \hline
\end{tabular}
\caption{Jazyková analýza, Nemecký jazyk}
\end{table}